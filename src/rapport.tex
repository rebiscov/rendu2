\documentclass[10pt,a4paper]{article}
\usepackage[utf8]{inputenc}
\usepackage{amsmath}
\usepackage{amsfonts}
\usepackage{amssymb}
\usepackage{graphicx}
\title{Projet 2: Rapport sur l'interpréteur fouine}
\author{Julien Du Crest, Vincent Rébiscoul}
\date{Pour le 16 Mai 2017}
\begin{document}
\maketitle

\section{L'idée}
Le but était de faire un interpréteur d'un langage très similaire à OCaml, Fouine. Nous avons implémenté tous ce qui était demandé pour le niveau ``intermédiaire''. Le niveau intermédiaire n'étant pas trop complexe, tous ce que nous avons fait est venu assez naturellement, ce qui fait que le rendu est probablement très classique. Cependant, je vais tout de même me permettre d'expliquer certains points. Premièrement, nous avons stocké nos objets (références, variables et fonctions) dans des tables de hachage. En effet, ce choix est logique car cette structure fait exactement dont nous avions besoin. Les tables de hachages se comportent comme des piles lorsque l'on ajoute plusieurs objets du même nom. Deuxièmement, lorsque l'on définit une fonction, on calcul sa clôture en parcourant la fonction et en ajoutant toute occurrence de variable ou de fonction dans une table de hachage que l'on va ensuite stocker. Ensuite, nous avons implémenté les exceptions à l'aide d'une référence vers un tuple global, c'est un tuple mutable qui contient le numéro de l'exception et un booléen indiquant si il y a eu une exception. 



\section{Les clôtures de fonction}
Ce fût un des points qui causa le plus de bogues. Premièrement, il fallait faire la différence entre les fonctions récursives et les fonctions classiques. Ce problème fut facilement levé, lorsque l'on définit une fonction récursive, il suffit, lors de la création de la clôture, d'ajouter la fonction à sa propre clôture. 

\end{document}