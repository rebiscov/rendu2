

\documentclass{beamer}
\usepackage[english]{babel}

\usetheme{default}
\usecolortheme{seahorse}
\usefonttheme[onlylarge]{structurebold}
\setbeamerfont*{frametitle}{size=\normalsize,series=\bfseries}
\setbeamertemplate{navigation symbols}{}

\usepackage[latin1]{inputenc}
\usepackage{times}
\usepackage[T1]{fontenc}

\definecolor{kugreen}{RGB}{25,93,25}

%\logo{\includegraphics[width=2cm]{logo_ens}}
%\useoutertheme{infolines} 

\addtobeamertemplate{navigation symbols}{}{%
	\setbeamercolor{footline}{fg = black}    
    \usebeamerfont{footline}%
	    
    \usebeamercolor[fg]{footline}%
    \hspace{1em}%
    \insertframenumber/\inserttotalframenumber
    \quad
    
    
}

\title{Proj II: la Fouine}
\author{Vincent Rebiscoul \& Julien du Crest}
\institute{ENS-Lyon}
\date{9 April 2017}



\begin{document}





\frame{\titlepage \vspace{-0.5cm}
}



\frame{
\frametitle{blabla}

}

\frame{
\frametitle{La theorie du pigeon superstitieux}
\begin{itemize}
	\item Periode 1 (j - 2 mois) : interpretation des fonctions a la definition, pas besoin de cloture.
\pause
	\item Periode 2 (j - 1 mois) : Table de hachage contenant les clotures des fonctions
\pause
	\item Periode 3 (j - 2 jour): Lazy evaluation
\pause
	\item Periode 4 (j - 1 jour): Vers une interpretation correcte?
\end{itemize}

}

\frame{
\frametitle{Conclusion}
\begin{itemize}
\item Il qu'il reste a ameliorer: 
recuperer toutes les erreurs trouvees par les TD-mens lors du test intensif (que nous n'avous pas fait nous meme) et les corriger. 
\item Ce  qui pourrait etre ajoute:
tout ce qu'il y a dans ocaml et qu'il n'y a pas ici.
\end{itemize}
}


\end{document}